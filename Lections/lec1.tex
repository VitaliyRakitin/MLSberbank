\documentclass[a4paper,14pt]{article}
%%%%%%%%%%%%%%%%%%%%%%%  ПАКЕТЫ  %%%%%%%%%%%%%%%%%%%%%%%%%%%%%%%%%%%%%%%%%%%%%%
\usepackage{cmap}                               % Чтобы в PDF работал человеческий поиск
\usepackage[X2,T2A]{fontenc}                    % T2A = русская кодировка. X2 = яти
\usepackage[utf8]{inputenc}                     % Ввод в универсальной кодировке
\usepackage{setspace,soulutf8}      		        % Чтобы можно было менять межстрочный и межбуквенный интервалы
\usepackage{amsmath,amsfonts,amssymb,amsthm}    % Символы для математики
\usepackage{mathrsfs}                           % Символы для математики
\usepackage{dsfont}                             % Шрифт для полей действительных, комплексных... чисел командой \mathds
\usepackage[subfigure]{tocloft}  % Многоточие в оглавлении
\usepackage{array,multicol,multirow,bigstrut}   % Чтобы можно было делать в таблице колонки фиксированной ширины, слитные ячейки, вставлять strut'ы.
\usepackage{indentfirst}                        % Абзацный отступ везде
\usepackage[british,russian]{babel}             % Русские переносы, тире, типографика, самодержавие, духовность!
\usepackage[perpage]{footmisc}                  % Сброс счётчика сносок на каждой странице
\usepackage[pdftex,unicode,bookmarks=true,bookmarksopen=true,colorlinks=true,linkcolor=blue,urlcolor=blue,citecolor=blue]{hyperref} % Синие ссылки в PDF
\usepackage{microtype}                          % Свешивающаяся пунктуация и подгонка белого пространства по правилу \pm 2 процента
\usepackage{textcomp}                           % Чтобы в формулах можно было русские буквы писать через \text{}
\usepackage[paper=a4paper,top=12.7mm, bottom=12.7mm,left=12.7mm,right=12.7mm,bindingoffset=6.6mm,includefoot]{geometry} % Достаточно экономные размеры листа и поля для нумерации страниц внизу (для колонтитулов в стиле лекций по матану, нужно includefoot заменить на includehead и не только
\usepackage{xcolor}                             % Чтобы можно было цветные объекты вставлять
\usepackage[pdftex]{graphicx}                   % Чтобы вставились изображения
\usepackage{float,longtable}                    % Поддержка плавающих таблиц и рисунков
\usepackage[margin=0pt,font=small,labelfont=bf,labelsep=period]{caption} % Подписи таблиц и рисунком мелкие, жирные, с принятым в русской типографике разделителем.
\usepackage{rotating}                           % Создание своих акцентов, поворот объекта.
\usepackage{datetime}                           % Отображение времени
%\usepackage{embedfile}                         % Чтобы код LaTeXа включился как приложение в PDF-файл
\usepackage{xspace}
\usepackage{wrapfig,enumitem}                   % Обтекаемые текстом рисунки
\usepackage{mathtools}                          % В тексте используется smashoperator, чтобы избежать некрасивых пробелов вокруг сумм и пределов с большим подстрочником
\usepackage{cancel}                             % Красивое <<вычёркивание>> сокращающихся выражений одноимённой командой
\usepackage{tikz,pgfplots}			% Рисование графиков непосредственно кодом
\usepackage{subfigure}
\usepackage{fancyhdr}	% Пакет для создания колонтитулов в стиле лекций по матану
\usepackage{accents}
\usepackage{pb-diagram}

%%%%%%%%%%%%%%%%%%%%%%%  ПАРАМЕТРЫ  %%%%%%%%%%%%%%%%%%%%%%%%%%%%%%%%%%%%%%%%%%%
\setstretch{1}                          % Межстрочный интервал
\flushbottom                            % Эта команда заставляет LaTeX чуть растягивать строки, чтобы получить идеально прямоугольную страницу
\righthyphenmin=2                       % Разрешение переноса двух и более символов
\pagestyle{plain}                       % Нумерация страниц снизу по центру.
\settimeformat{hhmmsstime}              % Формат времени с секундами
\widowpenalty=300                       % Небольшое наказание за вдовствующую строку (одна строка абзаца на этой странице, остальное --- на следующей)
\clubpenalty=3000                       % Приличное наказание за сиротствующую строку (омерзительно висящая одинокая строка в начале страницы)
\setlength{\parindent}{1.5em}           % Красная строка.
\setlength{\topsep}{0pt}                % Уничтожение верхнего отступа, если он где проявится
%%%%%%%%%%%%%%%%%%%%%%%%%%%%%%%%%%%%%%%%%%%%%%%%%%%%%%%%%%%%%%%%%%%%%%%%%%%%%%%

%%%% Техническая подготовка для определения следующих команд %%%%
\makeatletter
\newcommand*{\relrelbarsep}{.386ex}
\newcommand*{\relrelbar}{%
  \mathrel{%
    \mathpalette\@relrelbar\relrelbarsep
  }%
}
\newcommand*{\@relrelbar}[2]{%
  \raise#2\hbox to 0pt{$\m@th#1\relbar$\hss}%
  \lower#2\hbox{$\m@th#1\relbar$}%
}
\providecommand*{\rightrightarrowsfill@}{%
  \arrowfill@\relrelbar\relrelbar\rightrightarrows
}
\providecommand*{\leftleftarrowsfill@}{%
  \arrowfill@\leftleftarrows\relrelbar\relrelbar
}
\providecommand*{\xrightrightarrows}[2][]{%
  \ext@arrow 0359\rightrightarrowsfill@{#1}{#2}%
}
\providecommand*{\xleftleftarrows}[2][]{%
  \ext@arrow 3095\leftleftarrowsfill@{#1}{#2}%
}
\makeatother



%%%%%%%%%%%%%%%%%%%%% Команды-сокращения %%%%%%%%%%%%%%%%%%%%%%%%%%%%%%%%%%%%%%
\newcommand{\pau}{\hskip .75em plus.1em minus.08em\relax}	% пробел для выражений с кванторами, например «\forall\ \e>0\pau\exists\ \delta\colon», где \e определяется как
\newcommand{\ds}{\displaystyle}			% Быстрое переключение на выключный стиль формулы, где интегралы и суммы большие
\newcommand{\e}{\varepsilon}                % эпсилон
\newcommand{\p}{\partial}                   % частная производная
\renewcommand{\phi}{\varphi}                % Чтоб фи писалась в соответствии с русской традицией
\newcommand{\q}{\varnothing}			% Пустое множество в русской традиции
\newcommand{\N}{\mathds{N}}			% Натуральные числа
\newcommand{\Z}{\mathds{Z}}			% Целые числа
\newcommand{\Q}{\mathds{Q}}			% Рациональные числа
\newcommand{\R}{\mathds{R}}			% Действительные числа
\renewcommand{\C}{\mathds{C}}			% Комплексные числа
\newcommand{\T}{\mathbb{T}}			% Отмеченное разбиение (для интегральных сумм Римана или Римана"--~Стилтьеса
\newcommand{\Rim}{\mathcal R}			% Риман
\renewcommand{\le}{\leqslant}           % Правильное меньше или равно
\renewcommand{\leq}{\leqslant}           % Правильное меньше или равно
\renewcommand{\ge}{\geqslant}           % Правильное больше или равно
\renewcommand{\geq}{\geqslant}           % Правильное больше или равно
\newcommand{\dd}{\setminus}		% Ассоциируется с diagdown, но это именно вычитание множеств
\renewcommand{\iff}{\,\Leftrightarrow\,}	% Если и только если
\newcommand{\imp}{\hspace{1pt plus1pt}\Rightarrow\hspace{1pt plus1pt}}		% Следовательно
\def\hm#1{#1\nobreak\discretionary{}{\hbox{$#1$}}{}} % Команда для переноса на следующую строку символов бинарных операций
\renewcommand{\a}{\langle} 		% Открывающая треугольная скобка
\newcommand{\s}{\rangle}		% Закрывающая треугольная скобка
\newcommand{\ol}[1]{\overline{#1}}	% Надчёркивание аргумента по всей ширине

%%%% СТРЕЛКИ %%%%%%%%%%%%%%
\renewcommand{\to}{\rightarrow}                   % Правильная стрелка вправо («стремится») без возможности подписать
\newcommand{\To}[1]{\xrightarrow{#1}}		% Стрелка «стремится», ширина которой зависит от ширины агрумента-подписи (подпись ставится вверху)
\newcommand{\tend}[1]{\xrightarrow[#1]{}}	% Стрелка «стремится», ширина которой зависит от ширины агрумента-подписи (подпись ставится внизу)
\newcommand{\Tot}[1]{\xrightarrow{\text{#1}}}	% Стрелка «стремится», ширина которой зависит от ширины текстового агрумента-подписи (подпись ставится вверху)
\newcommand{\te}[1][]{\xrightarrow[n\to\infty]{#1}}	% Стремится при n стремящемся к бесконечности
\newcommand{\TE}{\xrightarrow[N\to\infty]{}}	% Стремится при N стремящемся к бесконечности
\newcommand{\neV}{\overline{\vphantom{<}\quad}\kern-.35em\searrow}	% Не возрастает в стиле лекций по анализу
\newcommand{\neU}{\underline{\vphantom{<}\quad}\kern-.35em\nearrow}	% Не убывает
\newcommand{\rsh}[2][P]{\xrightrightarrows[{#2\to\infty}]{#1}}		% Двойная растяжимая стрелка «равномерно сходится последовательность». Снизу стрелки подпись-обязательный-аргумент, к которому приписывается автоматически стремление к бесконечности. Необязательные аргумент изменяет подпись сверху с P на то, что напишите
\newcommand{\rsH}[2][X]{\xrightrightarrows[{#2}]{#1}} % Равномерная сходимость по параметру. Подпись снизу задаётся явно и обязательно; подпись сверху задаётся необязательно, по умолчанию X
\newcommand{\nsh}[2][n]{\xrightarrow[#1\to\infty]{\|\ \|_{#2}}}		% Растяжимая стрелка «сходится по норме». Вид нормы подписывается в обязательном аргументе (например l_2). Необязательный аргумент «что стремится к бесконечности» (стремление к бесконечности приписывается автоматически)


%%%%% Предел как оператор %%%%%%%%
\newcommand{\yo}[2]{\lim\limits_{#1\to#2}}	% предел{при этой букве}{стремящейся сюда}
\newcommand{\prb}[1]{\lim\limits_{#1}}		% предел{по такой базе}

%%%%%% O-символика %%%%%%%%%%%%
\newcommand{\oo}{\overline{\overline o}} 		% o-малое без подписей и скобок (две черты сверху ставятся автоматически). Переопределение этой команды изменит вид всех надстроек, указанных ниже
\newcommand{\ou}{\underline{\underline{O}}}		% Аналогично O-большое
\newcommand{\ouu}[2]{\underset{#2\ }\ou(#1)}		% O-большое{по отношению к чему}{при каком процессе}
\newcommand{\ooo}[2]{\underset{#2\ }\oo(#1)} % o-малое{по отношению к чему}{при каком процессе}
\newcommand{\ooob}[2]{\underset{#2}{\oo\big(#1\big)}}	% Аналогично, только скобки вокруг первого аргумента с плавающим размером
\newcommand{\ooog}[2]{\underset{#2}{\ou\big(#1\big)}}

%%%%%% Дифференцирование %%%%%%%%%%%%%%
\newcommand{\CP}[2]{\frac{\partial #1}{\partial #2}}	% Частная производная{чего}{по чему}
\newcommand{\cP}[2]{\frac{d #1}{d #2}}			% Полная/материальная производная
\newcommand{\Jacoby}[4]{\begin{pmatrix}
\CP {{#1}_{1}}{{#2}_1} 	& \CP{{#1}_{1}}{{#2}_2} & \dots & \CP{{#1}_{1}}{{#2}_{#3}} \\[1ex]
\CP {{#1}_{2}}{{#2}_1} 	& \CP{{#1}_{2}}{{#2}_2} & \dots & \CP{{#1}_{2}}{{#2}_{#3}} \\
\vdots 			& \vdots		& \ddots& \vdots \\
\CP {{#1}_{#4}}{{#2}_1} & \CP{{#1}_{#4}}{{#2}_2}& \dots & \CP{{#1}_{#4}}{{#2}_{#3}}	
\end{pmatrix}}% матрица Якоби{чего}{по чему}{размерность образа}{размерность аргумента}


\newcommand{\ttilde}[1]{\tilde{\tilde{#1}}}	% Набрать символ(ы)-аргумент с двумя узкими волнами
\newcommand{\Til}[1]{\widetilde{#1}{} }		% Набрать символ(ы)-аргумент с волной, ширина которой равна ширине аргумента

%%%% ПОСЛЕДОВАТЕЛЬНОСТИ, СУММЫ, ПРОИЗВЕДЕНИЯ %%%%%
\newcommand{\pos}[2][n]{\big\{{#2}_{#1}\big\}_{#1=1}^\infty} 		% Последовательность: в фигурных скобках аргумент с нижним индексом по умолчанию n; справа от скобок подпись n (или необязательный аргумент) от 1 до бесконечности
\newcommand{\ar}[4]{\big\{{#1}_{#2}\big\}_{#2=#3}^{#4}} % Задание последовательности четырьмя аргументами: что, каким символом нумеруется, откуда, докуда
\newcommand{\Har}[4]{\left\{{#1}_{#2}\right\}_{#2=#3}^{#4}} % Аналог предыдущего, но размер фигурных скобок подстраивается под внутренность.
\newcommand{\AR}[3]{\big\{#1\big\}_{#2}^{#3}} 		% Последовательность. Три аргумента: {что и как нумеруеся}{номер=начальное значение}{конечное значение номера}
\newcommand{\ry}[3][1]{\sum\limits_{{#3}={#1}}^\infty {#2}_{#3}}	% Ряд[с какого номерая начиная]{чего}{как нумеруется}; по умолчанию с единицы; Например, \ry an
\newcommand{\rY}[2][1]{\sum\limits_{{#2}={#1}}^\infty} % Сумма до бесконечности[с какого номера начать]{символ индекса}; например, \rY n или \rY[0]j
\newcommand{\RY}[3]{\sum\limits_{#1 = #2}^{#3}}		% Сумма{по этому индексу}{от этого значения индекса}{до этого значения индекса}; Например, \RY n1N
\newcommand{\tmy}[3][1]{\prod\limits_{#3=#1}^{\infty}#2_{#3}}	% Бесконечное произведение (от слова times), аналог \ry
\newcommand{\tmY}[2][1]{\prod\limits_{#2=#1}^{\infty}} % аналог \rY
\newcommand{\TMY}[3]{\prod\limits_{#1=#2}^{#3}}		% аналог \RY



%%%%% Окружения для нумерованных перечней
\newenvironment{iItems}{\begin{enumerate}\let\AEtheenumi\theenumi{}\renewcommand{\theenumi}{\roman{enumi}}\renewcommand{\labelenumi}{(\theenumi)}}{\renewcommand{\labelenumi}{\theenumi.}\renewcommand{\theenumi}{\AEtheenumi}\end{enumerate}}	% (i), (ii), ...
\newenvironment{azItems}{\begin{enumerate}\let\AEtheenumi\theenumi{}\renewcommand{\theenumi}{\asbuk{enumi}}\renewcommand{\labelenumi}{(\theenumi)}}{\renewcommand{\labelenumi}{\theenumi.}\renewcommand{\theenumi}{\AEtheenumi}\end{enumerate}}	% (а), (б), ...
\newenvironment{roItems}{\begin{enumerate}\renewcommand{\labelenumi}{(\theenumi)}}{\renewcommand{\labelenumi}{\theenumi.}\end{enumerate}}	% (1), (2), ...
\newenvironment{oItems}{\begin{roItems}\setcounter{enumi}{-1}}{\end{roItems}}	% (0), (1), ...
%%% В КОНЦЕ ФАЙЛА ПРИВЕДЕНЫ КОМАНДЫ, КОТОРЫЕ Я НЕ УСПЕЛ ОПИСАТЬ

%%%%%%%%%%%%%%%%%%%% Операторы в смысле LaTeX %%%%%%%%%%%%%%%%%%%%%%%%%%%%%%%%%
\DeclareMathOperator{\sign}{sgn}	% Знак (перестановки)
\DeclareMathOperator{\sgn}{sgn}		% Сокращённая альтернатива sign
\DeclareMathOperator{\diag}{diag}	% Диагональная матрица
\DeclareMathOperator{\const}{const}	% Константа
\DeclareMathOperator{\rang}{rank}         % Оператор ранга
\DeclareMathOperator{\rank}{rank}	% На любой вкус
\DeclareMathOperator{\diam}{diam}	% Диаметр, например, разбиения
\DeclareMathOperator{\card}{card}	% Мощность множества
\DeclareMathOperator{\osc}{osc}		% Осцилляция	
\DeclareMathOperator{\grad}{grad}	% Градиент
\DeclareMathOperator{\intS}{int}	% Внутренность множества
\DeclareMathOperator{\ext}{ext}		% Внешность множества
\DeclareMathOperator{\supp}{supp}


%%%%%%% Математическая экзотика %%%%%
\newcommand{\defequiv}{\mathbin{\vbox{\baselineskip=1.95pt\lineskiplimit=0pt\hbox{.}\hbox{.}\hbox{.}}\hskip-0.3em\equiv}} 		% вертикальное троеточие =, то есть «по определению тождественно»
\newcommand{\rus}[1]{\mbox{\rm{\scriptsize{#1}}}}	% команда убивает зависимость русского текста в формуле от стиля абзаца (но и от стиля формулы тоже). Специально для обозначения левой и правой производных, чтобы  f'_п печаталось с прямой буквой «п» и нельзя было спутать с n
\newcommand{\overcirc}[1]{\accentset{\circ}{#1}}	% круг над символом
\newcommand{\overstar}[1]{\accentset{\bigstar}{#1}}	% звезда над символом

%%%%%%%%%%%%%%%%%%% Глобальные текстовые команды %%%%%%%%%%%%%%%%%%%%%%%%%%%%%%
\newcommand{\ENGs}[1]{\foreignlanguage{british}{#1}} % Переключение правил переноса и прочих правил автооформления на английский язык действует на аргумент
\newcommand{\ENG}{\selectlanguage{british}} % Глобальное переключение (если просто начать писать на другом язык, велика вероятность появления очень больших пробелов или помещения части слова за границу листа
\newcommand{\RUS}{\selectlanguage{russian}}
\newcommand{\fnnsp}{\hspace{-0.4em}} % Знак сноски принято ставить до всех знаков препинания, кроме ! ? ...
% А если есть возможность подвинуть низкий знак препинания под сноску, то для этого Андрей Викторович и придумал \fnnsp
\let\myfootnote\footnote
\renewcommand{\footnote}[1]{\myfootnote{\;#1}} % Сноски Андрея Виторовича

%%%%%%%% Определение разрядки разреженного текста и задание красивых и притом регулируемых многоточий
% Зачем и почему описано в блоге http://kostyrka.ru/blog
\newdimen\ellipsiskern
\setlength{\ellipsiskern}{.1em}
\newdimen\ellipsiskernen
\setlength{\ellipsiskernen}{.2em}
\newcommand{\ldotst}{.\kern\ellipsiskern.\kern\ellipsiskern.}	% вместо ... пишем \ldotst{}
\newcommand{\ldotse}{!\kern\ellipsiskern.\kern\ellipsiskern.}	% вместо !.. пишем \ldotse{}
\newcommand{\ldotsq}{?\kern\ellipsiskern\kern-.11em.\kern\ellipsiskern.}	% вместо ?.. пишем \ldotsq{}
\newcommand{\ldotsten}{.\kern\ellipsiskernen.\kern\ellipsiskernen.}		% аглийское ...
\newcommand{\ldotspen}{.\kern\ellipsiskernen.\kern\ellipsiskernen.\kern\ellipsiskernen\kern.15em.}
\newcommand{\ldotseen}{.\kern\ellipsiskernen.\kern\ellipsiskernen.\kern\ellipsiskernen\kern.15em!}
\newcommand{\ldotsqen}{.\kern\ellipsiskernen.\kern\ellipsiskernen.\kern\ellipsiskernen\kern.067em?}
%%%%%%%%%%%%%%%%%%%%%%%%%%%%%%%%%%%%%%%%%%%%%%%%%%%%%%%%%%%%%%%%%%%%%%%%%%%%%%%



%%%%%%%%%%%%%%%%%%%%%%% ДРУГИЕ ПОЛЕЗНЫЕ КОМАНДЫ БЕЗ КОММЕНТАРИЕВ (ПОКА) %%%%%%%
\newcounter{rad}
\setcounter{rad}{1}
% a_1+a_2+\ldots
	\def\rad#1#2{\ifnum \therad < \the\numexpr(#2 + 1)\relax #1_\therad + \addtocounter{rad}{1} \rad{#1}{#2}\else \ldots \fi \setcounter{rad}{1}}
%a_{N+1}+a_{N+2}+\ldots
	\def\raD#1#2#3{\ifnum \therad < \the\numexpr(#2 + 1)\relax #1_{#3{\therad}} + \addtocounter{rad}{1} \raD{#1}{#2}{#3}\else \ldots \fi \setcounter{rad}{1}}
\def\crad#1#2#3#4{\ifnum \therad < \the\numexpr(#4 + 1)\relax #1{\therad}#3 + \addtocounter{rad}{1} \crad{#1}{#2}{#3}{#4}\else \ldots +#1{#2}#3 \fi \setcounter{rad}{1}}
\def\craD#1#2#3#4#5#6{\ifnum \therad < \the\numexpr(#4 + 1)\relax #1{#5{\therad}#6}#3 + \addtocounter{rad}{1} \craD{#1}{#2}{#3}{#4}{#5}{#6}\else \ldots +#1{#5#2#6}#3  \fi \setcounter{rad}{1}}
%repeat argument #1 #2 times
	\def\tfT#1#2{\ifnum \therad < \the\numexpr(#2 + 1)\relax #1\addtocounter{rad}{1} \tfT{#1}{#2}\else \fi \setcounter{rad}{1}}
\def\dI#1#2#3#4#5#6{\ifnum \therad < \the\numexpr(#6 + 1)\relax {#1}_{\therad}{#3}_{\therad}#5\addtocounter{rad}{1} \dI{#1}{#2}{#3}{#4}{#5}{#6}\else \dots #5{#1}_{#2}{#3}_{#4}\fi \setcounter{rad}{1}}
\def\DI#1#2#3#4#5#6{\ifnum \therad < \the\numexpr(#6 + 1)\relax #1\therad#3\therad#5\addtocounter{rad}{1} \DI{#1}{#2}{#3}{#4}{#5}{#6}\else \dots #5#1#2#3#4\fi \setcounter{rad}{1}}
\def\iSum#1#2#3{\ifnum \therad < #3\relax {#1}_{\therad}{#2}_{\the\numexpr(#3 - \therad + 1)\relax}+ \addtocounter{rad}{1}\iSum{#1}{#2}{#3}\else {#1}_{\therad}{#2}_{1}\fi \setcounter{rad}{1}}
\def\tms#1#2{\ifnum \therad < \the\numexpr(#2 + 1)\relax #1_\therad \ifnum \therad < \the\numexpr(#2)\relax \cdot \else \fi\addtocounter{rad}{1} \tms{#1}{#2}\else \cdots \fi \setcounter{rad}{1}}
\def\ctms#1#2#3#4#5{\ifnum \therad < \the\numexpr(#4 + 1)\relax #1{#2\therad}#3 \ifnum \therad < \the\numexpr(#4)\relax \cdot \else \fi\addtocounter{rad}{1} \ctms{#1}{#2}{#3}{#4}{#5}\else \cdots #1{#2#5}#3\fi \setcounter{rad}{1}}
\newcommand{\cSize}{\footnotesize}
\newcommand{\cmt}[1]{\text{\cSize #1}}
\newcommand{\BIggl}[1]{\left#1\vphantom{\Bigg(\frac12\Bigg)^2}\right.}
\newcommand{\BIggr}[1]{\left.\vphantom{\Bigg(\frac12\Bigg)^2}\right#1}

\usepackage{caption}
\usepackage{pdfpages}
    
    
\begin{document}
    \tableofcontents

\newpage
\section{Леция 1.}
\subsection {Введение.}
Где применяется машинное обучение:
\begin{itemize}
    \item Цены на нефть (есть данные нужно предсказать дальнейшее развитие);
    \item Кредитный скоринг (есть человек, есть кредит --- надо понять давать ему кредит или нет. Предсказываем вероятность того, на сколько он хороший заёмщик);
    \item Предсказание температуры (диапазон значений);
    \item Таргетированная реклама (на основе предпочтений подбираем наиболее подходящие товары);
    \item Информационный поиск;
    \item Персонализация (наиболее интересные новости VK -- градиентный бустинг деревьев);
    \item Чат-боты (NN);
    \item Призма;
    \item Генерация изображений (GAN); \hyperlink{https://deepmind.com/blog}{https://deepmind.com/blog}
    \item Боты в играх (reinforcement learning);
\end{itemize}

{\it Замечание:} Анонсирован градиентный бустинг на 2-ом занятии.

{\it Замечание:} Мода на сеточки :D надо доказать людям, что заниматься не нейронками тоже прикольно.


\subsection{По сути}

Пусь у нас есть тренировачные данные кредитного скоринга:
\begin{itemize}
    \item Пол;
    \item Возраст;
    \item Возвращают кредит или нет;
\end{itemize} 

Хотим предсказать вернёт он нам кредит или нет.

{\it Определение.} Дано $X$ --- обучающая выборка (множество объектов), где $x \in X = \{x\}$ --- прецедент, модель ML, после чего {\bf модель} обучается --- получается алгоритм машинного обучения. Y -- множество допустимых ответов (target).

Данные необходимо проверять --- модель хочет понять на сколько хорошо модель <<понимает>> данные, которые она никогда не видела. Проверяем обобщающую способность данных.

{\it Определение.} $y^{*}: X \rightarrow Y$ --- {\bf целевая функция.} Имеем значения только на конечном наборе данных $x_i \in X:$ $y^{*}(x_i) = y_i$. Сопоставляем входным данным соответствующие ответы.

{\it Определение.} $a$ --- {\bf решающая функция (алгоритм)} --- обобщение функции $y^{*}$ на всё множество объектов.

\[
    X = X_{\mathrm{train}} \cup X_{\mathrm{test}}
\]

Делим выборку на обучающую + тестовую.

{\it Замечание} {\bf Kaggle.} Титаник.

{\it Определение.} {\bf Признак } (feature) объекта x --- это результа измерения характеристики объекта. Для $x_j = \{x_j^{1}, \dots, x_j^{n}\}$ --- признаковое описание объекта (вектор).

Можно генерировать новые признаки различными алгоритмами (feature engineering).

Очень круто выяснить какие features наиболее важны для модели.

{\bf Формализуем:} по выборке $X_{\mathrm{train}}$ построить решающую функцию которая хорошо приблежает $y^{*}$ как на $X_{train}$, так и на всём множестве $X$.

Чтобы решать задачу необходимо ввести функционал качества. Для этого необходимо ввести метрику и функцию ошибки. Решаем задачу оптимизации: ищем набор параметров в пространстве параметров такой, чтобы Максимизировать функционал качества, минимизировать функцию ошибок ($Loss$).

\subsection{Метрики качества }
\[
    MAE = \frac{1}{N} \sum\limits_{1}^{N} | y_i - y_i^{*}|
\]

\[ 
    MSE = \frac{1}{N} \sum\limits_{1}^{N}  (y_i - y_i^{*})^2
\]
\[
    accuracy = \frac{TP + TN}{TP + TN + FP + FN}    
\]
\[
    PRE = \frac{TP }{TP + FP}    
\] 
\[
    REC = \frac{TP}{TP + FN}    
\]
\[
    F = 2 \cdot \frac{PRE \cdot REC}{PRE+REC}
\]

\begin{itemize}
    \item TP --- 1 и алгоритм говорит, что 1;
    \item TN --- 0 и алгоритм говорит, что 0;
    \item FP --- 0 алгоритм говорит, что 1;
    \item FN --- 1 алгоритм говорит, что 0;
\end{itemize}
Для оценки качества работы алгоритма на каждом из классов по отдельности введем метрики precision (точность) и recall (полнота).

Precision можно интерпретировать как долю объектов, названных классификатором положительными и при этом действительно являющимися положительными, а recall показывает, какую долю объектов положительного класса из всех объектов положительного класса нашел алгоритм.

Почему нужны эти метрики: Если данных одного класса сильно больше, чем другого, то алгоритму проще обучиться всегда выводить первый класс (вероятность ошибки ниже при accuracy). А precision и recall смещают данные.

\section{Машинное обучение}
\begin{itemize}
    \item Классическое обучение (с учителем -- без);
    \item Ансамблевые методы (много моделек);
\end{itemize}

{\it Определение.}  Классификация. есть множество объектов и множество классов (конечное). Обучаем модель, разделяющую данные на группы.

{\it Определение.}  Регрессия. отличается тем, что допустимым ответом является число или числовой вектор (пространство непрерывно, бескоечное множество ответов).

Без учителя:
\begin{itemize}
    \item Кластеризация;
    \item Поиск правил;
    \item Уменьшение размерности;
\end{itemize}


{\it Определение.} {\bf Переобучение } --- модель находит зависимости там, где их нет (например у всех плохих заёмщиков были красные ботинки);
Причины:
\begin{itemize}
    \item Слишком много степеней свободы (слишком большая модель);
    \item Мало данных (general зависимости не находим, а находим локальные зависимости);
\end{itemize}

{\it Определение.} Валидационная выборка --- знаем на ней ответы (на тестовой не знаем);

{\it Определение.} Кросс-валидация --- ошибка на валидационных данных не отображает реальной картины мира.
Делим N раз на валидационную и тестовую, N раз тренируем алгоритм, считаем среднюю ошибку.

{\bf Методы борьбы}

{\it Регуляризация} (добавляем функцию от весов в loss, сдерживаем их рост)

\newpage
\section{ Лекция 2. Модели, ансамбли моделей }
    \subsection{Зоопарк моделей}
    Занимаемся моделями обучения с учителем.

    Есть 2 варианта постановки задачи:
    \begin{itemize}
        \item Классификация;
        \item Регрессия;
    \end{itemize}

    \subsubsection{Наивный байесовский классификатор}
    Самый простой способ, как построить себе модель.

    {\bf Замечание.} { \it Только классификация.}

    Для класса определяем вероятность. Есть условная вероятость, как часто встречается фича с фиксированной величиной для каждого класса.

    По формуле Байеса найдём вероятность ответа при значении фичей.
    \[
        p(A|B) = \frac{p(A) \cdot p(B|A) }{ p(B) }
    \]

    {\bf Определение.} { \it Априорная вероятность.} -- это вероятность, присвоенная событию при отсутствии знания, поддерживающего его наступление $p(A)$;

    {\bf Определение.} { \it Апостериорная вероятност.} -- это условная вероятность события $p(A|B)$.


    \begin{itemize}
        \item Spam detection;
        \item Сегментация новостных статей по теме;
        \item Определение эмоционального окраса текста;
    \end{itemize}

    \subsubsection{Линейные модели}

    {\bf Определение.} { \it Линейная модель.} выражается линейной функцией.
    \[
        Y_i = \beta_0 + \beta_1 f(X_{i1}) + \cdot + \beta_p f(X_{ip}) + \e_i
    \]
    Где $\e_i$ -- шум (гауссово распределение), $i = 1,\dots, N$.

    {\bf Замечание.} Функции $f_1, \dots, f_p$ не обязательно должны быть линейными.

    \subsubsection{Линейная регрессия}

    \[
        y = X \cdot w^{T} + \e
    \]

    В данном случае решение будет решением СЛАУ:
    \[
        w^T = \left(W^{T}W\right)^{-1}W^T y
    \]

    {\bf Проблема 1.} решение будет <<осмысленным>> только тогда, когда в данных реально есть линейная зависимость. В реальных данных почти не встречается.

    {\bf Проблема 2.} Линейная зависимость признаков, $\det W^TW$ должен быть отличен от $0$.

    Решение проблемы: {\bf регуляризация}:
    \[
        w^T = \left(W^{T}W + \lambda \cdot \mathrm{Id} \right)^{-1}W^T y
    \]

    \subsubsection{Градиентный спуск}

    Веса модели ищутся с помощью градиентного спуска
    \begin{enumerate}
        \item Инициализация $W$ случайными значениями;
        \item Итеративно для всех сэмплов $x \in X$:
        \begin{enumerate}
            \item Вычислить функцию потерь на $x$;
            \item Вычислить производные функции потерь $grad_w$ по каждому из $w \in W$;
            \item Обновляем веса: $w = w - \mathrm{lr} \cdot grad_w$
        \end{enumerate}
        \item Полученное $W = \{w\}$ --- искомое.
    \end{enumerate}

    Для линейной регрессии функция потерь:
    \[
        MSE = \frac{1}{N} \sum\limits_{i=1}^N \left( y_i - (W x_i + b)\right) ^ 2
    \]


\end{document}